\subsection{Literaturanalyse und -synthese}
Teile der gefundenen Literatur wurde zunächst aussortiert. Hierfür wurden im ersten Schritt Duplikate entfernt, worauf folgend im zweiten Schritt die Literatur auf harte Kriterien geprüft wurde. Kriterien, welche hier Anwendung fanden, waren:
\begin{itemize}
    \item \enquote{Peer Reviewed Literature} (Veröffentlichungen in Jorunals oder Proceedings)
    \item Monographie/Buch
    \item Whitepaper
    \item Dissertationen
    \item Sprache: Englisch
\end{itemize}
Im dritten Schritt wurde der Titel und das Abstract der gefunden Beiträge näher betrachtet, um dessen Relevanz für Transparenz im ML einschätzen zu können. So wurden einige Paper aussortiert, die nicht wirklich dem Thema dieser Masterarbeit zuzuordnen sind und beispielsweise nur deswegen in der Suche erschienen, weil sie den Einsatz von ML-Technologien untersuchten, um eine Domäne wie z.B. die Finanzbranche transparenter zu machen. Zu betrachtende Quellen nach dieser ersten Reduktion waren 296 Quellen für RQ1.

das wurde mit dem zweiten Suchtherm nicht mehr gemacht, da hier ja von Anfang an nur die relevanten mit neuen Informationen dokumentiert wurden

Um diese viele Suchergebnisse noch weiter zu strukturieren, wurden die Quellen zunächst noch in grobe Kategorien eingeteilt. Beiträge, welche Transparenz und ML im Titel oder im Abstract behandelten, wurden als relevant klassifiziert, wenn diese auch konkrete Methoden betrachteten. Mit mittlerer Relevanz wurden solche Themen eingeordenet, welche eher den Fokus auf andere Themen legten (z.B. ethische oder rechtliche Behandlungen) legten, aber transparenzschaffende Maßnahmen betrachteten. Mit niedriger Relevanz wurden solche Quellen gekennzeichnet, welche Transparenz im Zusammenhang ML nur als Beispiel betrachteten. Sehr niedrig, wenn sie nur kurz erwähnten, dass etwas transparent ist. Zur inhaltlichen Abgrenzung kamen noch weitere Kategorien hinzu, welche sich zum Teil bereits im Vorhinein aus der Konzeptualisierung in Kapitel \ref{chap:Konzeptualisierung} ergaben, jedoch entstanden auch einige Kategorien neu nach Überfliegen der Literatur. So wurden beispielsweise Quellen identifiziert, welche nur bestimmte XAI-Methoden behandelten oder solche, die einen ganzheitlichen Ansatz zur Transparenz im ML darlegten oder sich nur auf Daten fokussierten.

\todo{Bei direkten Zitaten Seitenzahl? Joshua fragen}