\subsection{Definition des Umfangs der Literaturrecherche}
Um eine Abgrenzung des Umfangs einer Literaturrecherche vorzunehmen, empfehlen \cite{vom_Brooke_2009} die Taxonomie von \cite{cooper_organizing_1988}, welche in Abbildung \ref{tab:taxonomyCooper} dargestellt ist. Je nach Charakteristika kann sich während einer Literaturrecherche auf unterschiedliche Kategorien fokussiert werden. Die Entscheidungen, wo jeweils der Fokus liegen sollte, ergab sich meist aus der Aufgabenstellung. In der Abbildung \ref{tab:taxonomyCooper} ist durch Fettschrift hervorgehoben, welches Merkmal im Fokus lag, was folgend kurz begründet wird.

\begin{table}[h]
\begin{tabular}[h]{p{3.5cm}|p{8cm} l|l}
\large\textbf{Charakteristika} & \large\textbf{Kategorien} \\
\hline
Fokus & \begin{itemize}[nosep, leftmargin=*, before=\vspace{-0.5\baselineskip},after =\vspace{-\baselineskip}]
\item \textbf{Forschungsergebnisse}
\item Forschungsmethoden
\item Theorien
\item \textbf{Praktiken oder Anwendungen}
\end{itemize} \\
\hline
Ziel & \begin{itemize}[nosep, leftmargin=*, before=\vspace{-0.5\baselineskip},after =\vspace{-\baselineskip}]
\item \textbf{Integration}
\item Kritik
\item Identifikation zentraler Herausforderungen
\end{itemize} \\
\hline
Perspektive & \begin{itemize}[nosep, leftmargin=*, before=\vspace{-0.5\baselineskip},after =\vspace{-\baselineskip}]
\item \textbf{Neutral}
\item Bewertend
\end{itemize} \\
\hline
Umfang & \begin{itemize}[nosep, leftmargin=*, before=\vspace{-0.5\baselineskip},after =\vspace{-\baselineskip}]
\item Vollständig
\item Vollständig mit Selektion der Zitierungen
\item \textbf{Repräsentativ}
\item Zentral
\end{itemize} \\
\hline
Organisation & \begin{itemize}[nosep, leftmargin=*, before=\vspace{-0.5\baselineskip},after =\vspace{-\baselineskip}]
\item Historisch
\item \textbf{Konzeptuell}
\item Methodologisch
\end{itemize} \\
\hline
Zielgruppe & \begin{itemize}[nosep, leftmargin=*, before=\vspace{-0.5\baselineskip},after =\vspace{-\baselineskip}]
\item Spezialisierte Wissenschaftler
\item Generalisierte Wissenschaftler
\item \textbf{Praktiker}
\item Gesellschaft
\end{itemize} \\
\end{tabular}
\caption{\label{tab:taxonomyCooper} Umfang der Literaturrecherche, angelehnt an \cite{cooper_organizing_1988}}
\end{table}

\begin{enumerate}[label={(\arabic*)}]
    \item \textbf{Fokus:} Da eine Handreichung, welche sich im praktischen Kontext einsetzen lassen soll, primärer Gegenstand der Masterarbeit ist, legen eher Praktiken und Anwendungen im Fokus. Nichtsdestotrotz sollen auch Ergebnisse wissenschaftlicher Forschungen in die Gestaltung der Handreichung einfließen, da diese Handreichung im Rahmen einer wissenschaftlichen (Master-)arbeit entsteht.
    \item \textbf{Ziel:} Ziel der Literaturrecherche ist die Integration und Zusammenfassung verschiedener Methoden und Ansätze, welche in der Literatur existieren, um ein ML-System transparenter zu gestalten.
    \item \textbf{Perspektive:} Bei dem Zusammentragen verschiedener Maßnahmen wird eher eine neutrale Position eingenommen. Eine Wertung erfolgt jedoch (teilweise) in der Evaluation dieser Arbeit zu einem späteren Zeitpunkt.
    \item \textbf{Umfang:} Da in der Masterarbeit mehrere Forschungsfragen beantwortet werden müssen, in denen teilweise auf transparenzschaffende Maßnahmen mehrere ML-Verfahren eingegangen werden muss, wurde sich gegen eine vollständige Erfassung jeder zur Verfügung stehenden Literatur entschieden, um den zeitlichen Rahmen nicht zu sprengen. Stattdessen wurde eine repräsentative Recherche angestrebt, welche auf selektierten wissenschaftlichen Datenbanken aufbaut.
    \item \textbf{Organisation:} Die Rechercheergebnisse werden konzeptuell dargestellt.
    \item \textbf{Zielgruppe:} Die Ergebnisse der Literaturrecherche dienen vor allem der Erstellung der Handreichung, welche sich per Aufgabenstellung eher an Praktiker, wie z.B. die Entwickler von ML-Systemen richten soll.
\end{enumerate}