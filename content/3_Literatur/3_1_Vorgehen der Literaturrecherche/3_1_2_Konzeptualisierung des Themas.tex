\subsection{Konzeptualisierung des Themas}
\label{chap:Konzeptualisierung}
Als zweiten Schritt sehen \cite{vom_Brooke_2009} vor, sich einen groben Überblick über das Thema zu verschaffen, um potenziell relevante Bereiche zu identifizieren. Hierfür wurde eine unstrukturierte Literaturrecherche durchgeführt, um Kernkonzepte zu finden, welche später den Grundstein für die strukturierte Literaturrecherche bilden, da durch diese relevante Suchbegriffe identifiziert werden konnten. Einige Kernaspekte standen nichtsdestotrotz durch die Aufgabenstellung bereits fest, aber um sich zusätzlich einen Überblick zu verschaffen, wurde der Suchbegriff \enquote{Transparency in Machine Learning} in die Datenbank Google Scholar eingegeben. 

In der folgenden Tabelle \ref{tab:suchbegriffe} sind die die Ergebnisse der Konzeptualisierungsphase, also die Suchbegriffe der strukturierten Literaturrecherche abgebildet. Nach \cite{Brink.2013} sind hierbei auch noch Synonyme, Ober- und Unterbegriffe sowie verwandte Begriffe miteinzubeziehen.

\begin{table}[!h]
\begin{tabular}[h]{p{4cm}|p{10cm} l|l}
\large\textbf{Gruppierung} & \large\textbf{Kernbegriffe} \\
\hline
Maschinelles Lernen & Machine Learning/ML, Künstliche Intelligenz/KI, Artificial Intelligence/AI, Erklärbares Maschinelles Lernen/Explainable Artificial Intelligence/XAI, Black-Box \\
\hline
Transparenz & Transparency, Verständlichkeit/Erklärbarkeit/Explainability, Interpretierbarkeit/Interpretability, DSGVO/GDPR, Zurechenbarkeit/Accountability \\
\hline
Methoden & Entscheidungsbäume/Decision Trees/Random Forest, k-means Clustering, Support Vector Machines, Logistische Regression/Logistic Regression, k-nächste Nachbarn/k-Nearest Neighbours, Bayessche Netze/Bayesian Networks, (Künstliche) Neuronale Netze/Artificial Neural Network/Deep Learning \\
\hline
Prozess & Process, Entwicklungslebenszyklus/Lebenszyklus/Entwicklung/Development Lifecycle/Lifecycle, Daten/Data, Datensatz/Dataset, Pipeline, Fluss/Flow \\
\end{tabular}
\caption{\label{tab:suchbegriffe}Suchbegriffe}
\end{table}

Um der Literaturrecherche eine auf die Forschungsfragen orientierte Struktur zu geben, wurden spezielle Suchterme definiert, welche teilweise individuell auf die jeweiligen Datenbanken zugeschnitten wurden und im Anhang zu finden sind. Während der Erstellung wurden die jeweiligen Begriffe zunächst probeweise in die Datenbank IEEE \footnote{https://ieeexplore.ieee.org/Xplore/home.jsp} eingegeben, um ungefähr abzuschätzen, welche und wie viele Ergebnisse erzielt werden. 

\textbf{Grundsätzliche transparenzschaffende Maßnahmen (RQ1)}
\begin{addmargin}[25pt]{0pt}
In Bezug auf die erste Forschungsfrage (\textit{RQ1: Wie können die unterschiedlichen Ebenen einer ML-Pipeline transparent gemacht werden?}) wurde zunächst versucht alle Begriffe der Konzeptualisierung in den Suchterm miteinzubauen. Jedoch brachte allein der Suchterm \enquote{Transparency AND ML OR DATA} bereits 1.338.968 Ergebnisse, was als unüberschaubar erschien. Aufgrund dieser sehr vielen Suchergebnisse wurde sich dafür entschieden, den Fokus auf den thematischen Kern dieser Masterarbeit zu legen, sodass für die erste Forschungsfrage der Suchterm \enquote{Transparency AND Machine Learning} gewählt wurde, da dieser auch schon zu genügend Ergebnissen führte.
\end{addmargin}

\textbf{Transparenzschaffende Maßnahmen konkreter ML-Verfahren (RQ2)}
\begin{addmargin}[25pt]{0pt}
Da der Fokus bei der zweiten Forschungsfrage (\textit{RQ2: Welche Ansätze existieren, um verschiedene ML-Algorithmen, wie z.B. Entscheidungsbäume oder auch neuronale Netzwerke zu erklären?}) eher auf den konkreten Methoden, welche innerhalb des ML-Lebenszyklusses eingesetzt werden liegt, wurde sich dazu entschieden, nach Möglichkeiten zu suchen, diese transparent, aber auch erklärbar zu machen. Der Aspekt des erklärbaren maschinellen Lernens wurde hier deswegen miteinbezogen, da in Kapitel \ref{chap:Transparenz_Definition} bereits herausgestellt wurde, dass Erklärbarkeit und Transparenz zusammenspielen. Somit wurde für den zweiten Forschungsaspekt nach dem jeweiligen Verfahren sowie nach Transparenz und Methoden der XAI gesucht. Für Entscheidungsbäume lautete der Suchterm beispielsweise: \enquote{decision tree AND Transparency AND XAI}.
\end{addmargin}

\textbf{Transparenzschaffende Maßnahmen je Zielgruppe}
\begin{addmargin}[25pt]{0pt}
In Bezug auf die dritte Forschungsfrage (\textit{RQ3: Inwieweit und auf welche Weise sollten ML-Algorithmen transparent gemacht werden, wenn diese Experten oder fachfremden Personen veranschaulicht werden?}) konnte während der Konzeptualisierung wenig konkretes gefunden werden, was als Suchterm sinnvoll erschien. In Absprache mit dem Betreuer dieser Masterarbeit wurde sich dazu entschieden, auf diese Forschungsfrage etwas weniger Gewicht zu legen als auf die anderen beiden. Inhaltlich wurden jedoch die Ergebnisse, welche sich mit dem anderen Suchthermen finden ließen, auch auf diese dritte Forschungsfrage hin geprüft.
\end{addmargin}