\subsection{Erklärungen}
XAI- oder Visualisierungsmethoden können unterschiedliche Arten von Erklärungen produzieren. 

\cite{hernandez2021explainable} haben in einer Studie in der Luftfahrtbranche untersucht, welche Art von Erklärungen sich am besten eignen, um Vertrauen zu schaffen. Folgende Auflistung stellt die Ergebnisse da, wobei solche Erklärungen, welche am ehesten Vertrauen in die Technologie schaffen, in absteigender Reihenfolge genannt werden:
\begin{itemize}
    \item Wichtigste Faktoren, welche die Vorhersage beeinflussen
    \item Konkrete Beispiele für die Vorhersage
    \item Visuelle Darstellung der Funktionsweise
    \item Kausale Erklärungen (Welche Änderungen an den Eingabedaten oder am Algorihmus selbst würde eine andere Vorhersage ergeben?)
    \item Erläuterung des Lebenszyklus und des Entwurfs
    \item Kontrafaktische Erklärung (Warum ist die Vorhersage \enquote{A} und nicht \enquote{B}?)
\end{itemize}

Wenngleich \cite{hernandez2021explainable} herausgefunden haben, dass sich zumindest für das Vertrauen kontrafaktische Erklärungen weniger eignen, stellen \cite{tsiakmaki2021case} heraus, dass diese Art der Erklärungen Erkenntnisse liefern können, welche bei der praktischen Umsetzung während der Programmierung hilfreich sein können. Inbesondere liefern kontrafaktische Erklärungen ein gutes Verständnis über die Ursache-Wirkung-Beziehung.
