\subsection{Modellauswahl}
Wie vorangegangen beschrieben existieren auch im ML-Bereich transparente Methoden. Um ein ML-Modell somit transparent zu machen, läge es also nahe einfach ein transparentes zu wählen, jedoch gibt es für diesen Ansatz Pro- und Contraargumente. 
Wird ein interpretierbares Modell gewählt, stößt man eventuell auf Grenzen, welche sich besonders bei nichtlinearen, komplexen Zusammenhängen oder einem Problemraum mit besonders vielen Merkmalen in den Trainingsdatensätzen zeigen \cite{hanif2021survey}.
Andere Autoren sehen jedoch in der Praxis eine andere Herangehenweise als zielführend an. Einfache Modelle funktionieren nicht unbedingt schlechter in der Realität. Vor allem in Branchen wie der Medizin oder Strafverfolgung, in denen Menschen das System gut verstehen sollten, kann es wünschenswert sein eine höhere Interpretierbarkeit als eine besonders gute Genauigkeit zu implementieren \cite{vaughan2020human, keller2020augmenting}.