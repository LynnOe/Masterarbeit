\subsection{Definition}
\label{chap:Transparenz_Definition}
Der Begriff Transparenz kommt in den unterschiedlichsten Domänen vor, meint grundsätzlich jedoch, dass etwas durchschaubar oder nachvollziehbar ist \cite{DudenTransparenz}. Im ML-Umfeld hilft Transparenz dem in Kapitel \ref{subsec:XAI} beschriebenen Blackbox-Charakter entgegen zu wirken und lässt sich im Datenschutz-Umfeld \enquote{Klarheit, Erkennbarkeit und Nachverfolgbarkeit} verstehen \cite{bedner2010schutzziele, rudin2019stop}. Konkret sollten \enquote{Systeme [...] durchschaubar und ihre Funktions- und Arbeitsweise nachvollziehbar und verständlich} sein und die verarbeiteten Daten sowie die beteiligten Personen und Handlungen offenlegen \cite{bedner2010schutzziele}. Im XAI-Umfeld finden sich noch weitere Definitionen und Einordnungen des Begriffes der Transparenz. \cite{kamath2021explainable} stellen heraus, dass ein transparentes ML-Modell sowohl eine verständliche Struktur als auch einen verständlichen Algorotihmus besitzen muss. Transparenz ist eine Voraussetzung dafür, Modelle und deren Funktionsweisen zu verstehen. So sei an dieser Stelle erwähnt, dass Transparenz mit Erklärbarkeit (vgl. die Definition im vorangegangenen Kapitel \ref{subsec:XAI}) eng verwoben ist. Die Beziehung zwischen diesen Konzepten ist derart, dass sich Erklärbarkeit positiv auf Transparenz auswirkt \cite{kamath2021explainable}.