\subsection{Relevanz von Transparenz bei Algorithmen des maschinellen Lernens}
Transparenz ist insbesondere bei Algorithmen des MLs wichtig, da diese über einen stark ausgeprägten Black-Box-Charakter verfügen \cite{rudin2019stop}. Es kann jedoch gewünscht sein, nachzuvollziehen, wie und warum ein Algorithmus, eine Entscheidung getroffen hat \cite{fink2020quick}. Für diesen Wunsch nach Transparenz kommen verschiedene Motive unterschiedlichster Akteure in Frage. So ist es denkbar, dass es einem Entwickler wichtig ist, das System zu verstehen mit dem schlussendlichen Ziel dessen Funktionsweise zu verbessern. Daneben ist es für die Nutzer einer Software bedeutsam das System zu verstehen, um Vertrauen in die Technologie fassen zu können. Auch während der Nutzung eines Systems ist es von Vorteil, wenn das Verhalten des Systems nachvollziehbar bleibt, denn dies führt auch dazu, dass der Nutzer nicht zu einem Konkurrenzprodukt greift, was wiederum dem Softwarehersteller zugute kommt. Aber nicht nur für unmittelbar Systembetroffene ist Transparenz von Bedeutung, auch für die Gesellschaft ist es bedeutsam Stärken und Grenzen von Technologien einordnen zu können \cite{weller2019transparency}. Des weiteren rückt Transparenz in den Vordergrund, wenn Menschen fürchten, die Kontrolle über das eingesetzte System zu verlieren, was besonders im medizinischen Kontext fatale Folgen haben könnte \cite{fink2020quick, antoniadi2021current}. Aber auch im strafrechtlichen Umfeld, in dem Personen auf Grundlage von Entscheidungen eines Algorithmus Überwachung zu befürchten haben, wird Transparenz als ethisch und rechtlich notwendiges Gut betrachtet.Ferner ist auch ein nichtdiskriminierender Charakter gewünscht, welcher die Einhaltung der Grundrechte sichert \cite{EUVorschlag}. Auch lassen sich Systemverantwortliche oder das gesamte System mittels Transparenz leichter überwachen und in Bezug auf Sicherheitsstandards testen \cite{weller2019transparency, singh2018decision}. Überwachung des Systems ist z.B. bei selbstfahrenden Autos denkbar, in dem im Falle eines Unfalls Zurechenbarkeit und rechtliche Haftung ermöglicht werden soll \cite{weller2019transparency}.  Auch in einem Vorschlag der Europäischen Kommission \cite{EUVorschlag} findet sich die Forderung nach Transparenz besonders bei bestimmten Systemeinsatzkontexten (z.B. Deep-Fake-Systeme) wieder, um Manipulationsrisiken erkennen zu können. 

Die Forderung nach Transparenz ist jedoch nicht nur ein Motiv einiger Akteure, sondern bereits in bestehende Gesetze etabliert. Eine Forderung nach Transparenz bei der Datenveranverarbeitung findet sich in der Datenschutz-Grundverordnung (DSGVO) wieder. In Art. 15 DSGVO werden Privatpersonen beispielsweise die Rechte eingeräumt, Auskunft über die \enquote{Verarbeitungszwecke} sowie \enquote{die Kategorien personenbezogener Daten, die verarbeitet werden} einfordern zu dürfen. Die DSGVO fordert insbesondere in Artikel 12, dass \enquote{alle Informationen und [...] Mitteilungen [...], die sich auf die Verarbeitung beziehen, in präziser, transparenter, verständlicher und leicht zugänglicher Form in einer klaren und einfachen Sprache [übermittelt werden sollen]}.

Wenngleich viele, berechtigte Gründe für Transparenz innerhalb von ML-Systemen existieren, sei an dieser Stelle angemerkt, dass dies auch negative Folgen nach sich ziehen kann. So ist es möglich, dass Transparenz zu Lasten von Effizienz eingeführt wird, aber es ist auch denkbar, dass Transparenz eines Systems ausgenutzt wird, um ebendieses zu manipulieren oder die daraus gewonnenen Informationen unangemessen zu nutzen \cite{weller2019transparency}.