Nach \cite{konigstorfer2021software} sind Dokumentation ein gutes Werkzeug zur Kommunikation und lassen sich gleichzeitig auch dazu einsetzen, ein System transparent, fair sowie zurechenbar zu machen. Sollen IT-Systeme Menschen näher gebracht werden, ist zunächst festzustellen, wer angesprochen werden soll. Im Umfeld der SW-Architektur-Dokumentationen stellen eine Art von Dokumentationen solche dar, welche für zukünftige Teammitglieder geschrieben sind, die aus dieser Motivation heraus das System verstehen wollen. Eine solche Dokumentation bietet darüber hinaus das Potenzial das System zu analysieren und zu verbessern. Neben der technischen Seite, sollten weiter verschiedene Stakeholder wie Nutzer, Auftraggeber oder Projektmanager beachtet werden, mit denen das System kommuniziert werden soll. Neben dem \emph{Wer} gibt es bei der Kommunikation von Systemen die Frage nach dem \emph{Was}. Grundsätzlich können einzelne Module, welche Teile von Software abbilden, Gegenstand von Dokumentationen sein. Auch kann die Performance z.B. als Fluss-Diagramm abgebildet werden. Weiterführend gibt es Unterschiede in der Art der Kommunikation. Diese ist informal durch die Unterstützung von Grafiken möglich, aber auch semiformal z.B. in Form eines UML-Diagramms \cite{bass2003software}. Die Standards im Bereich der Softwaredokumentation lassen sich nur bedingt auf selbstlernende Algorithmen übertragen \cite{rodvold1999software}. Im Rahmen der Handreichung in Kapitel tbd wird auf Besonderheiten bei Verfahren des maschinellen Lernens eingegangen.
\todo{Vlt. hier noch auf Besonderheiten bei ML eingehen?}