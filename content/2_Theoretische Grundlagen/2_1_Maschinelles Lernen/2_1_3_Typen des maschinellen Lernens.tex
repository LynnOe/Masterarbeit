\subsection{Typen des maschinellen Lernens}
\label{chap:Types}
Einzelne Algorithmen des MLs lassen sich zumeist einer von mehreren Klassen zuordnen. Hauptunterscheidungsmerkmal dieser Kategorien besteht in der Art des Lernens sowie in der Beschaffenheit der Ein- und Ausgabedaten. Die drei am häufigsten genannten Klassen sind überwachtes Lernen (engl.: \emph{supervised learning}), unüberwachtes Lernen (engl.: \emph{unsupervised learning}) und bestärkendes Lernen (engl.: \emph{reinforcement learning}), wobei auch noch Mischformen wie teilüberwachtes Lernen (engl.: \emph{semi-supervised learning}) existieren \cite{Alpaydin+2019, Wuttke.2022, ayodele}.

\textbf{Überwachtes Lernen}
\begin{addmargin}[25pt]{0pt}
Überwachtes Lernen ist der am häufigsten genutzte Typ des maschinellen Lernens. Hierbei lernt ein Computerprogramm, indem es sich bestehende Daten zu Nutze macht \cite{Verdhan.2020}. Diese zu Beginn vorhandenen Datensätze enthalten immer einen Input und einen Output \cite{Alpaydin+2019, Choo.2020}. Ein Algorithmus soll lernen, wie Merkmale, welche sich in den Daten befinden, mit den Zielvariablen zusammenhängen \cite{Wuttke.2022}. Es soll also eine Funktion $f:x \rightarrow y$ entstehen, welche folgende Input-Paare $\{(x\textsubscript{1}, y\textsubscript{1}), ... , (x\textsubscript{i}, y\textsubscript{i})\}$  aufeinander abbildet. Schlussendlich sollen Zielvariablen ($y$-Werte) für neue, unbekannte $x$-Werte mit der Funktion vorhergesagt werden \cite{Choo.2020}.

Typisch für Probleme, welche mithilfe von überwachtem Lernen angegangen werden, sind Klassifikations- und Regressionsprobleme\todo{Später überdenken: eigenes Kapitel für Anwendungsfälle sinnvoll?}. Klassifikationsalgorithmen beispielweise ermöglichen es auf Bildern zu erkennen, ob es sich um eine Katze oder einen Hund handelt. Die im vorherigen Abschnitt angesprochene Zielvariable y wäre in diesem Fall ein Wert, welcher einer Kategorie, also Hund oder Katze, entspräche. Ein Beispiel für ein Regressionsproblem ist die Vorhersage von Regentagen auf Grundlage historischer Niderschlagsmengen \cite{Choo.2020}. 
\end{addmargin}

\textbf{Unüberwachtes Lernen}
\begin{addmargin}[25pt]{0pt}
Im Gegensatz zum überwachten Lernen stehen beim unüberwachtem Lernen keine Trainingsdaten mit Zielvariablen zur Verfügung \cite{ayodele}. Das Ziel bleibt jedoch, dass eine Funktion erstellt werden soll, die Eingabe- auf Ausgabewerte $f:x \rightarrow y$ abbildet, jedoch ohne Kenntnisse über $y$ \cite{Alpaydin+2019}. Algorithmen des unüberwachten Lernens funktionieren so, dass sie den Daten Informationen extrahieren und folglich Muster in den Daten entdecken \cite{Choo.2020, Verdhan.2020}.

Beispiele für die Anwendung von unüberwachtem Lernen sind Clusteranalysen oder Assoziationsalanysen \cite{Wuttke.2022}. Bei einer Clusteranalyse werden Daten aufgrund ähnlicher Muster oder gleicher Attribute gruppiert. Muster können dadurch entstehen, dass z.B. bei Entitäten eines Clusters die gleichen Merkmale vorhanden sind oder fehlen. Als praxisnaher Anwendungsfall sei hier die Kundensegmentierung genannt, welche es einem Unternehmen ermöglicht verschiedene Marketingstrategien passend für unterschiedliche Kundensegemente zu entwickeln. Merkmale, die im Rahmen einer Clusteranalyse ins Gewicht fallen, wären z.B. das Alter, der Umsatz oder das Online/Offline-Verhalten \cite{Verdhan.2020}.
Assoziationsanalyse als weiteres Beispiel des unüberwachten Lernens haben zum Gegenstand Assoziationsregeln zu finden, die Zusammenhänge zwischen verschiedenen Attributen darstellen, wie es z.B. in der Warenkorbanalyse der Fall ist. Hier wird versucht, anhand von bestimmten im Warenkorb befindlichen Produkten auf andere Produkte zu schließen \cite{Cleve.2020}. Eine Regel, welche ein Algorithmus der Assoziatitonsanalyse aufstellen würde, könnte beispielsweise lauten: \textit{Wer Müsli kauft, der kauft auch Milch}.
\end{addmargin}

\textbf{Teilüberwachtes Lernen}
\begin{addmargin}[25pt]{0pt}
Das teilüberwachte Lernen bildet die Mischform des über- und unüberwachten Lernens, denn hier werden sowohl Daten mit Zielvariablen als auch Daten ohne Zielvariablen zum Trainieren des Algorithmus verwendet \cite{ayodele}. Teilüberwachtes Lernen ist meistens dann sinnvoll, wenn einige Daten mit Zielvariable, jedoch gleichzeitig auch viele ohne Zielvariable vorhanden sind und es sehr teuer oder aufwendig wäre, die fehlenden Zielvariablen zu erheben \cite{Wuttke.2022}. Die Grundidee des teilüberwachten Lernens ist, dass Daten, welche sich im gleichen Cluster befinden, wahrscheinlich auch zur gleichen Klasse gehören. Diese Erkenntnis kann genutzt werden, um eine bessere Datenbasis zu generieren. Somit können beide ML-Ansätze kombiniert werden \cite{Verdhan.2020}.
\end{addmargin}

\textbf{Bestärkendes Lernen}
\begin{addmargin}[25pt]{0pt}
Das bestärkende Lernen beschäftigt sich damit, wie mehrere Aktionen aufeinander folgen sollen. Ein Algorithmus lernt basierend auf Belohnungen oder Bestrafungen, welche durch seine Umwelt signalisiert werden, wie er zu handeln hat \cite{Lorenz.2020}. Dieses Feedback ist zwingend erforderlich, um selbstständig lernen zu können \cite{Wuttke.2022}. Ziel ist es mit  zukünftigen Aktionsreihenfolgen die Belohnungen zu maximieren \cite{Lorenz.2020}. Weiter ist nicht eine einzige Aktionen, sondern mehrere aufeinander folgende Aktionen zusammengefasst als Taktik, entscheidend für die angemessene Funktionsweise des Algorithmus \cite{Alpaydin+2019}. 

Ein prominentes Beispiel für bestärkendes Lernen ist die von Google entwickelte Software \textit{AlphaGO}, welche mittels Datensammlungen über von Menschen gespielter Partien, geeignete Taktiken entwickeln konnte, um gegen den erfolgreichsten menschlichen Spieler im  Brettspiel GO gewinnen zu können \cite{Wuttke.2022}.
\end{addmargin}

Je nach dem, welcher Typ zur Anwendung kommt, entstehen Differenzen bzgl. des technischen Ablauf des MLs, welcher im vorherigen Kapitel \ref{subsec_AblaufML} dargestellt wurde. Beim überwachten Lernen wird dem Lernprozess die Teilung der Beispieldaten vorangestellt. Der Datensatz wird hier in einen Trainings- und ein Testdatensatz aufgeteilt. Diese Testdaten werden genutzt, um das erstellte Modell auf seine Funktionsweise und inhaltliche Zuverlässigkeit zu testen \cite{kotsiantis2007supervised}. Es ist für die Funktionsweise des Algorithmus und die Performance während des Trainings jedoch nicht unerheblich, wie groß diese Datensätze im Verhältnis sind \cite{nguyen2021influence}. Auch beim bestärkenden Lernen kommen Testdaten zum Einsatz \cite{Lorenz.2020}. Da die Trainingsdaten, welche beim unüberwachten Lernen vorliegen, über keine Zielvariablen verfügen, kann die Überprüfung des erzeugten Modells nicht wie beim überwachten Lernen mithilfe von Testdaten erfolgen \cite{ayodele}. Alternativen, um das Modell bei diesen Algorithmen zu überprüfen, bestehen beispielsweise in Metriken, welche überprüfen, wie kompakt die verschiedenen Cluster erstellt wurden, oder in Kreuzvalidierung \cite{halkidi2001clustering, perry2009cross}.  

An dieser Stelle sei angemerkt, dass neben der Funktionweise auch andere Qualitätsmerkmal bei ML eine Rolle spielen können. So kann die Performance oder Nutzung des Hardwarespeichers auch entscheidend bei der Implementierung eines ML-Algorithmus sein \cite{coleman2019analysis}. Daneben existieren noch weitere Methoden zur Evaluation im ML, welche es z.B. ermöglichen, verschiedene Instanzen des gleichen Algorithmus oder auch unterschiedliche Algorithmen untereinander zu vergleichen \cite{raschka2018model}.